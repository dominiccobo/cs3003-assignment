\section{Dataset}

% What was the dataset chosen.

\begin{table}[H]
	\begin{tabular}{ | l |  l |}
		\hline
		Project Name: & Jenkins \\ 
		\hline
		Git VCS Upstream: & https://github.com/jenkinsci/jenkins\\
		\hline
	\end{tabular}
\end{table}

% what does the implementing code of the dataset do....

\subsection{What is Jenkins ?}

Jenkins started off as a do-it-all automation tool as an alternative to CRON. It has evolved to become one of the most flexible, extensible continuous integration and delivery platforms in industry.

\subsection{Why Jenkins ?}

The extremely extensible nature of Jenkins suggests that as a tool it is likely to be very maintainable; however, a certain notoriety follows Jenkins around when talking about Jenkins and asserting its behaviour. 

Newer products in the market have taken to limiting extensibility in favour of standardisation and convention; this is suggestive of a potential issue with extreme modifiability and lack of testability.

\subsection{Collected Metrics}

%TODO : run collection of data.