\section{Methodology}

The latest mainline branch of an open source software system was cloned from the GitHub Git Version Control System. A set of tools was then used to analyse and measure the dataset for the below.

\begin{description}
	\item [Code-smell frequency] 
	
	Measurement the number of code smells and their frequency across methods, classes and packages.
	
	\begin{table}[H]
		\begin{tabular}{ | l | l | l |}
			\hline
			Imperative Abstraction & Multifaceted Abstraction & Unnecessary Abstraction \\
			\hline
			Unutilised Abstraction & Deficient Encapsulation & Unexploited Encapsulation \\
			\hline
			Broken Modularization & Broken Hierarchy & Cyclic Hierarchy \\
			\hline
			Cyclic-Dependent Modularization & Insufficient Modularization & Hub-like Modularization \\
			\hline
			Deep Hierarchy & Missing Hierarchy & Multipath Hierarchy \\ 
			\hline
			Wide Hierarchy & Abstract Function Call From Constructor & Rebellious Hierarchy \\
			\hline
			Complex Conditional & Complex Method & Empty catch clause \\
			\hline
			Long Identifier & Long Method & Long Parameter List \\
			\hline
			Long Statement & Magic Number & Missing default \\
			\hline
		\end{tabular}
		\caption{Supported code smells by used tool}
		\label{tab:supportedCodeSmells}
	\end{table}
	
	\item [Attribute naming descriptiveness]
	
	Lexically parsing function, variable and class names with an evaluation of the correlative semantic similarities between each.
	
	\item [Test coverage] Test coverage provides a rough idea of how many statements are executed throughout a test suite. 
	
	This metric will likely suffer from the pitfalls that lines may very well be executed, but incorrect assertions may result in 'untested' code.
		
	\item [Lines of Production code vs Lines of Test Code Ratio] 
	
	Understanding the ratio between lines of production code and test code ratio should roughly signal the effort delivered to ensure functionality feedback can be easily obtained.
	
	This could be reinforced using an assertion density comparison.
	
	%TODO :	 read up on http://wiki.c2.com/?ProductionCodeVsUnitTestsRatio
	
	\item [Cyclomatic complexity] 
	
	A metric aimed at measuring the structuredness of the code by measuring the number of avenues of execution. 
	
	Tests should be clear and non-complex. Production code is likely to have a far higher complexity. Higher complexity should suggest the requirement for focusing on testability.
	
	% TODO Is code standards adherance / violation frequency also a viable venue to explore 
	
	\item [Maintainability Index (Control Benchmark)] As a control benchmark, the Maintainability index (including lines of comments) will be measured.
\end{description}